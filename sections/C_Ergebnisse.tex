		
% III. Ergebnisse und Diskussion
A. Präsentation der Ergebnisse
    Vergleich Performance original vs js-backend vs chez-backends
    Vergleich zu anderen Sprachen aus AWFY (?)
    
B. Diskussion der Ergebnisse im Kontext der Zielsetzung
    Wie hoch ist der Overhead den Effekt mitbringt
    Wieso sind manche Backends besser als andere (?)
    Wie nützlich ist FastEffekt in der entwicklung von effekt
        Performance
        Agiert als End to End test
    Wie stabil sind die Performance Ergebnisse auf verschiedenen Maschinen
    
C. Interpretation der Ergebnisse und deren Relevanz
    FastEffekt kann nur zwischen Backends des momentanen snapshots vergleichen
    kann nicht über viele effekt releases vergleichen (breaking changes)

D. Grenzen der Aussagekraft
    Kernfeatures von Effekt: effekthandling ist nicht teil der Benchmarks
    keine benchmarks ohne Effekt in den Backend sprachen: mlton, llvm
    A. Teststrategie und Testfälle
        - verifikation durch hardcodierte Werte der originalen Benchmarks
        - verifikation in Pipeline	
        note: first and last laufzeit sind verdächtig anders

Of course very important! You need to discuss the informatics as well as bio part of your thesis topic.

\subsection{ Technische Einschränkungen durch Effekt }
\subsubsection{ Import System }
Wie in 4. Implementierung %TODO reference
 beschrieben, ist jeder Microbenchmark ein eigenständiges Kommandozeilenwerkzeug, die Logik für das Parsen der Parameter und Aufrufen der Benchmarks befindet sich in runFromCli, eine Funktion die jeder Benchmark importiert und nutzt.
runFromCli ist eine Abweichung gegenüber der AWFY Implementation, die stattdessen eine \textit{Harness} Klasse vorsieht:
\begin{quote}
    Harness

    should be entry point for all executions
    dynamically selects benchmark to execute based on arguments

    https://github.com/smarr/are-we-fast-yet/blob/master/docs/guidelines.md
\end{quote}
Effekt hat nur ein sehr rudimentäres Import System. Wir können Module einzeln importieren, aber es werden immer alle Funktionen aus dem Modul importiert und wir können die importierten Funktionen nicht umbenennen.
Das bedeutet, dass es regelmäßig zu \textit{Nameclashes} kommt: Zwei Funktionen aus verschiedenen Modulen tragen den selben Namen und werden in die gleiche Datei importiert. Anschließend ist es für uns nicht mehr möglich beide zu verwenden, da immer nur eine Implementation mit diesem Namen im Namespace existieren kann.
\begin{lstlisting}[language=javascript]
----------------------------
// src/astropyhsics.effekt
module astropyhsics

def calculate(): Int = {
  10 + 2 * 24 - 7
}

-----------------------------
// src/economics.effekt
module economics

def calculate(): String = {
  "investments +45%"
}

def invest(): Int = {
    5
}
-----------------------------
// src/myProgram.effekt
import astropyhsics
import economics

def main() {
  //to use invest, we must import the whole economics module
  println(invest())

  //undefined behaviour because astropyhsics.calculate and economics.calculate clash
  println(calculate())
}
\end{lstlisting}

Eine gängige Möglichkeit das Problem zu umgehen, ist in anderen Sprachen, hier am Beispiel von TypeScript, der gezielte Import von Teilen eines Moduls, beispielsweise nur einzelner Funktionen. Alternativ können ganze Module importiert werden, aber die Funktionen behalten ihren Namespace.
Eine dritte Möglichkeit ist das Umbenennen von imports.

\begin{lstlisting}[language=javascript]
import { calculate } from "./astropyhsics";
calculate();

import { invest } from "./economics"
invest();

import { calculate as economicsCalculate} from "./economics";
economicsCalculate();

import * from "./economics" as economics;
economics.calculate();
\end{lstlisting}

Effekt bietet keine dieser Möglichkeiten. Die einzige Option ist ein aufwändiger Extraschritt. Wir erstellen ein neues Modul, importieren dort einen der beiden clashenden Imports und erschaffen für die von uns genutzten Funktionen neue Wrapperfunktionen, mit selber Signatur aber eineindeutigem Namen.
Dieses neue "Zwischen"Modul können wir nun ohne Nameclashes importieren.
Das muss für jeden Nameclash wiederholt werden, alternativ könnten wir als Entwickeler unsere Module so gestalten, dass alle Funktionsnamen global eineindeutig sind. Eine suboptimale Situation.
\begin{lstlisting}
-----------------------------
// src/intermediate.effekt
module intermediate

import economics

def investWrapper(): String = {
  invest()
}

//this module now only exports investWrapper, not other functions from economics.

-----------------------------
// src/myProgram.effekt
import astropyhsics
import intermediate

def main() {
  //to use invest, we must import the whole economics module
  println(investWrapper())

  //is now astropyhsics.calculate
  println(calculate())
}
\end{lstlisting}

Aus dieser Schwierigkeit mit Imports heraus, haben wir uns entschieden Imports generell weitestgehend zu vermeiden.
Die Benchmarks sind so aufgebaut, dass jeder Benchmark alleinstehend ausgeführt wird und kein Modul alle Benchmarks importiert. Stattdessen importieren die Benchmarks alle die für sie nötige Funktionalität, analog zur Nutzung einer Library. Die stellt eine Abweichung zur Originalimplementation aus AWFY dar, ist aber unserer Meinung nach geboten, da wir es nicht als zielführend erachten ein unfertiges Importsystem zu nutzen. Auch hier gilt wieder, dass die Änderung die Performanz nicht oder kaum beeinflussen sollte und deshalb unproblematisch ist.

In der Originalimplementation von AWFY wird nicht unterschieden zwischen miniBench und normalBench, ausserdem erben alle Microbenchmarks vom Benchmark Interface. Dabei existieren einige Codesmells wie Refused Bequest und das Ändern der Signatur geerbter Methoden, das gesamte Vererbungskonstrukt ist unserer Meinung nach suboptimal. Die objektorienterte Programmierung wird in Effekt nicht unterstützt und dient nur der COdequalität, beziehungsweise der Entwicklung. Sie beeinflusst weiterhin nicht die Performanz. Aus diesen Gründen haben wir uns entschieden von dem unsauber umgesetzen Vererbungskonstrukt abzuweichen.

Interfaces existieren in Effekt, sind aber sehr restriktiv: Es dürfen nur exakt die Funktionen implementiert werden, welche das Interface definiert. Da aber die Benchmarks für die interne Logik unterschiedliche Funktionen nutzen, müsste man das Interface ausserhalb der Logik definieren. Ein Extraschritt, der aus unserer Sicht keinen Vorteil bringt, denn:

Durch die invertierten Abhängigkeiten, stehen die Microbenchmarks als letztes Glied in der Kette der Abhängigkeiten. Es brächte deshalb keinen Vorteil hier die Benchmarks von einem Interface erben zu lassen, selbst wenn sie alle ähnliche Funktionen teilen, denn kein anderes Modul importiert und nutzt die Benchmarks. 
Aus diesen Gründen verzichten wir darauf, alle Microbenchmarks von einem gemeinsamen Interface erben zu lassen.
Weitere Abweichungen innerhalb der Benchmarks vom Original beschreiben wir individuell für jeden Benchmark.


\subsubsection{ Quality of Life }
Effekt ist eine funktionale Sprache. Are We Fast Yet wurde ursprünglich in Java implementiert, die als Vorlage ausgewählten Javascript Benchmarks nutzen auch stark objektorienterte Methoden.
So ist jeder Benchmark eine Klasse, welche vom Benchmark Interface erbt, ganz analog zur Java Implementation.
Diese Vorgehensweise ist in Effekt nicht möglich, Klassen und Objekte sind nicht implementiert, werden nicht unterstützt.

Interfaces sind möglich in Effekt, aber deutlich restriktiver: Es dürfen nur exakt die Member implementiert sein, welche vom Interface vorgegeben werden, nicht mehr und nicht weniger.
In Java und Javascript gibt es nur die Vorgabe, dass es mindestens die Member sein müssen, mehr sind aber erlaubt.

[...] //TODO


Auch Funktionsobjekte, welche in der allgemeinen Javascript Entwicklung viel genutzt werden, sind nicht möglich: Funktionen können nicht in Variablen "gespeichert" werden, können auch nicht von Funktionen returned werden.
Der einzige erlaubte Weg eine Funktion weiter zu geben, ist als Parameter einer Higher Order Funktion, tiefer in den Call Stack hinein.

\subsubsection{ Overload des Aktionspunktes }
Zwar sind aufrufe wie "meinInterface.machEtwas()" möglich, allerdings ist dies keine echtes Objekt Orientiertes paradigma, sondern nur Syntaktik Sugar.
Auch ist dieser Overload des "Aktionspunktes" so instabil, dass er kaum genutzt wird in FastEffekt. Eigentlich kann jede Funktion so aufgerufen werden, dass sie als Methode des ersten Parameters erscheint, 
doch der Compiler erkennt dies oft nicht an. Ein Großteil der Versuche dies zu benutzen führte zu Compiler Fehlermn, weshalb ich schnell davon Abstand genommen habe.

\subsubsection{ Zu schnelle Benchmarks }
Als Kompromiss zwischen Aussagekraft und kurzer Laufzeit des Tools wurde arbiträr eine Laufzeit von zwei Sekunden pro Benchmark in Effekt gewählt.
Mit zehn Microbenchmarks dauert ein kompletter Durchlauf von FastEffekt damit mindestens zwanzig Sekunden plus die Laufzeit der Javascript-AWFY-Vergleichsbenchmarks.
Gesteuert werden kann die Dauer eines Benchmark-Durchlaufs über die Problemgröße und über innere Iterationen. Ist ein Benchmark deutlich zu langsam, kann die Problemgröße reduziert werden, ist ein Benchmark deutlich zu schnell, kann die Anzahl innerer Iterationen künstlich erhöht werden. Damit bietet sich eine begrenzte Kontrolle über die Laufzeit an.

Wie bereits erwähnt, bilden die Javascript AWFY-Implementationen die Basis mit der die Performance der Effekt Benchmarks verglichen wird.
Da vorallem das Effekt-Javascript-Backend teils deutlich langsamer ist als die AWFY-Javascript Version, //TODO auf analyse section beziehen
 muss die Problemgröße der Benchmarks klein gehalten werden. Ansonsten überschreiten die Effekt Benchmarks die Zwei-Sekunden-Grenze, und benötigen bis zu 30 Sekunden. Wenn allerdings die Problemgröße gedrückt wird, fällt auch die Dauer der Javascript Benchmarks, bis unter 2 Millisekunden. Die Zeitauflösung der Benchmarks ist nicht feingranular genug, um unter 1 Millisekunde zu messen. Das führt zu geloggten Zeiten von 0,1 und 2 Millisekunden, für die meisten Benchmarks. 

Mit solchen grob granularen Zahlen lässt sich in der Analyse wenig anfangen. Mit einer Laufzeit von 0 Millisekunden wird sogar der Vergleich zur Effekt Laufzeit verfälscht und eine Ratio von positiv unendlich angegeben, oder Not a Number (NaN).

Eine Lösung wäre eine feinere Zeiterfassung in Javascript, Millis und Mikrosekunden. Allerdings ist hier zu beachten, dass möglicherweise trotz feinerer Messung wieder Schwankungen im Interpreter, Garbage Collection und Laden von Dependencies das Ergebnis stark beeinflussen können.

Eine alternative Lösung ist die Erhöhung der Problemgröße oder der inneren Iterationen, nur für Javascript, nicht aber für die Effekt Benchmarks.
Das hat den Vorteil dass die Javascript Laufzeit aussagekräftiger wird, aber die Vergleichbarkeit mit der Effekt Laufzeit sinkt.

Die dritte Lösung ist das Erhöhen von Problemgröße und inneren Iterationen in Effekt und Javascript, so bleibt Aussagekraft und Vergleichbarkeit erhalten.
Damit wird das FastEffekt Tool schmerzhaft langsam für den Endnutzer, wenn Laufzeiten von einer Minute und mehr erreicht werden, und verletzt damit das Kernrequirement eine schnelle Möglichkeit von Performance Messung zu bieten.

\subsubsection{ Technische Abweichungen der Microbenchmarks }
Hier wollen wir erläutern, an welchen Stellen wir abgewichen sind von der Originalen AWFY-Implementation und inwiefern dies das Ergebnis der Performanzmessungen beeinflussen könnte.

Implementiert in Effekt sind alle 9 Microbenchmarks aus Are-we-fast-yet:
\begin{itemize}
\item Bounce simulates a ball bouncing within a box.
\item List recursively creates and traverses lists.
\item Mandelbrot calculates the classic fractal. It is derived from the Computer Languages Benchmark Game.
\item NBody simulates the movement of planets in the solar system. It is derived from the Computer Languages Benchmark Game.
\item Permute generates permutations of an array.
\item Queens solves the eight queens problem.
\item Sieve finds prime numbers based on the sieve of Eratosthenes.
\item Storage creates and verifies a tree of arrays to stress the garbage collector.
\item Towers solves the Towers of Hanoi game.
\end{itemize}
