% I. Einleitung
\medskip
Diese Arbeit ist wie folgt strukturiert
\begin{itemize}
    \item In \cref{sec:mainideas} Wir stellen die Kernideen und Zielsetzung der Arbeit vor.
    \item In \cref{sec:technical-meat} Wir präsentieren die Ergebnisse und Analyse der Ergebnisse.
    \item In \cref{sec:discussion} Wir beschreiben die Implementation des Programms und technische Probleme die dabei aufgetreten sind.
\end{itemize}

\subsection{ Problemstellung }
Um die Sprache Effekt auf Performance hin zu untersuchen während der Entwicklung, sollen Benchmarks ausgeführt werden.
Die Ausführung der Benchmarks liefert numerische Werte welche die Laufzeit der einzelnen Benchmarks beschreiben. So können die Entwickler:innnen von Effekt
abschätzen ob eine Änderung an der Sprache selbst die Performance verbessert oder verschlechtert.
Als Grundlage werden die Benchmarks von Are-We-Fast-Yet verwendet, diese Benchmarks sollen in Effekt übersetzt und implementiert werden.
Der Fokus soll dabei auf dem Vergleich der Performance zwischen zwei Effekt Versionen liegen, und nicht um die Performance von Effekt mit anderen Sprachen zu vergleichen.

\begin{itemize}
%   TODO: hinfaken, high level, nicht technisch
    \item Benchmarks aus Are We Fast Yet
    \item sprachen interner Performance vergleich, nicht cross-language
    \item tool für development an Effekt
    \item möglichst direkt feedback zur performance änderung an entwickler
    \item unix basiert, nicht windows
    \item ziel backend ist JS
\end{itemize}

\subsection{ Zielsetzung der Arbeit}
\subsubsection{ Forschungsfragen }
\begin{itemize}
    \item Lassen sich klassische OOP Programme übersetzen nach Effekt, und wie
    \item Wie performant sind die verschiedenen Backends von Effekt
    \item Wie nutzbar ist Effekt / Wie kaputt ist Effekt
\end{itemize}


  